%%%%%%%%%%%%%%%%%
% This is based on an example CV created using altacv.cls (v1.1.4, 27 July 2018) written by
% LianTze Lim (liantze@gmail.com), based on the
% Cv created by BusinessInsider at http://www.businessinsider.my/a-sample-resume-for-marissa-mayer-2016-7/?r=US&IR=T
%
%% It may be distributed and/or modified under the
%% conditions of the LaTeX Project Public License, either version 1.3
%% of this license or (at your option) any later version.
%% The latest version of this license is in
%%    http://www.latex-project.org/lppl.txt
%% and version 1.3 or later is part of all distributions of LaTeX
%% version 2003/12/01 or later.
%%%%%%%%%%%%%%%%
\documentclass[10pt,a4paper,academicons]{altacv}

\usepackage[hidelinks]{hyperref}
\usepackage[super]{nth}

\geometry{left=1cm,right=9cm,marginparwidth=6.8cm,marginparsep=1.2cm,top=1cm,bottom=1cm}

\setmainfont{Lato}

\definecolor{VividPurple}{HTML}{3E0097}
\definecolor{SlateGrey}{HTML}{2E2E2E}
\definecolor{LightGrey}{HTML}{666666}
\colorlet{heading}{VividPurple}
\colorlet{accent}{VividPurple}
\colorlet{emphasis}{SlateGrey}
\colorlet{body}{LightGrey}

\renewcommand{\itemmarker}{{\small\textbullet}}
\renewcommand{\ratingmarker}{\faCircle}

\newcommand{\stackoverflowsymbol}{\faStackOverflow}
\newcommand{\stackoverflow}[1]{\printinfo{\stackoverflowsymbol}{#1}}

\newcommand{\dateofbirthsymbol}{\faBirthdayCake}
\newcommand{\dateofbirth}[1]{\printinfo{\dateofbirthsymbol}{#1}}

\newcommand{\homesymbol}{\faHome}
\newcommand{\home}[1]{\printinfo{\homesymbol}{#1}}

\addbibresource{main.bib}

\newif\ifpublic{}

\publictrue{}
%\publicfalse{}

\begin{document}

\name{Santiago Castro}
\tagline{PhD Student}
\photo{2.5cm}{me}
\personalinfo{%
  \email{\href{mailto:sacastro@umich.edu}{sacastro@umich.edu}}
  \dateofbirth{Sept. \nth{29}, 1991}
  \location{Ann Arbor, MI, USA}
  \home{Montevideo, Uruguay}
  \github{\href{https://github.com/bryant1410}{bryant1410}}
  \twitter{\href{https://twitter.com/bryant1410}{bryant1410}}
  \linkedin{\href{https://linkedin.com/in/santiagocastroserra}{santiagocastroserra}}
  \stackoverflow{\href{https://stackoverflow.com/story/bryant1410}{bryant1410}}
}

\begin{fullwidth}
\makecvheader{}

Diligent Computer Scientist with experience in Research and Development for both the industry and the academy. Passionate about NLP and ML.\@ Free Software fan.

\end{fullwidth}

\cvsection[page1sidebar]{Education}

\cvevent{Ph.D.\ in Computer Science}{University of Michigan}{Sept. 2018 --- Present}{Ann Arbor, MI, USA}

\begin{itemize}
  \item Advisor: Rada Mihalcea
\end{itemize}

\divider{}

\cvevent{B.E.\ in Computer Science}{Universidad de la República}{March 2010 --- May 2015}{Montevideo, Uruguay}

\begin{itemize}
  \item \nth{3} student that got graduated, chronologically, out of a generation of 665.

  \item GPA:\@ 9.9 (out of 12, average at the university is 5.6). \nth{4} in GPA ranking.

  \item Extracurricular courses: Introduction to Deep Learning, Deep Learning applied to Natural
  Language Processing, Formal Grammars for Natural Language, General-Purpose Computing on
  Graphics Processing Units and Semantic Parsing.
\end{itemize}

\cvsection{Work Experience}

\cvevent{Lead Machine Learning Engineer}{Xmartlabs}{September 2016 --- August 2018}{Montevideo, Uruguay}

\begin{itemize}
  \item I started a Machine Learning area, highly focused on Mobile Computer Vision projects. It included the pioneering idea, design and development leadership of the open source project Bender: a novel framework to run neural networks in iOS for real-time purposes, using the iPhone’s GPU.\@ This project, after its release, has subsequently seen the emerge competitors from big tech companies (e.g., Google’s TF Lite, Apple’s Core ML, Facebook’s Caffe2Go and Baidu’s MDL). Worked also on other CV+ML projects for the Entertainment, Retail and Agro industries.
  \item \emph{Bender}: \url{https://github.com/xmartlabs/Bender}
  \item \emph{Dreamsnap}: \url{https://getdreamsnap.com}
\end{itemize}

\divider{}

\cvevent{Web Developer}{Department of Social and Political Sciences, Universidad Católica del Uruguay}{June 2016 --- November 2016}{Montevideo, Uruguay}

\begin{itemize}
  \item Development of a web site for a project to show the network between political parties and the donors in their campaigns.
  \item \url{https://finpol.github.io}
\end{itemize}

\pagebreak

\cvsection[page2sidebar]{Work Experience (continued)}

\cvevent{Teaching \& Research Assistant (``Asistente --- Grado 2'')}{Universidad de la República}{March 2014 --- August 2018}{Montevideo, Uruguay}

\begin{itemize}
  \item \emph{Natural Language Processing Group} member.
  \item \url{https://www.fing.edu.uy/inco/grupos/pln/}
\end{itemize}

\divider{}

\cvevent{Lead Android Engineer}{Xmartlabs}{March 2014 --- August 2016}{Montevideo, Uruguay}

\begin{itemize}
  \item Mainly the development of Android apps (mainly in the projects \emph{Lynkos CRM}, \emph{Feeld} and \emph{Franklin}). Also worked in frontend development with HTML, CSS, Javascript and jQuery, and backend using Ruby, with the Sinatra and Rails frameworks.
  \item Maintenance of a local server with several deployments. Docker implementation in several projects.
\end{itemize}

\divider{}

\cvevent{Co-founder \& IT Manager}{Yojuego}{July 2012 --- Present}{Montevideo, Uruguay}

\begin{itemize}
  \item Side-project that gives a space to players from basketball youth academies from Uruguay. The site presents videos, photos, stats and media coverage for the subject.
  \item Management of a Virtual Private Server, Apache web server, MySQL database system and Wordpress web site. Also worked on the product development.
  \item \url{https://yojuego.com.uy}
\end{itemize}

\cvsection{Publications}

\nocite{*}

%\printbibliography[heading=pubtype,title={\printinfo{\faBook}{Books}},type=book]

%\printbibliography[heading=pubtype,title={\printinfo{\faFileTextO}{Journal Articles}}, type=article]

\printbibliography[heading=pubtype,title={\printinfo{\faGraduationCap}{Theses}},type=thesis]

\printbibliography[heading=pubtype,title={\printinfo{\faGroup}{Conference Proceedings}},type=inproceedings]

\printbibliography[heading=pubtype,title={\printinfo{\faGroup}{Workshop Proceedings}},type=incollection]

\printbibliography[heading=pubtype,title={\printinfo{\faAsterisk}{Non Refereed}},type=misc]

\cvsection{Talks}

\cvevent{A High Coverage Method for Automatic False Friends Detection for Spanish and Portuguese}{VarDial 2018 @ COLING 2018}{August 2018}{Santa Fe, New Mexico, USA}

\divider{}

\cvevent{A Crowd-Annotated Spanish Corpus for Humor Analysis}{SocialNLP 2018 @ ACL 2018}{July 2018}{Melbourne, Australia}

\divider{}

\cvevent{Mobile Machine Learning}{Machine Learning undergraduate course, Universidad de la República}{May 2018}{Montevideo, Uruguay}

\divider{}

\cvevent{Mobile Machine Learning with Bender}{Deep Learning Applied to Computer Vision graduate course, Universidad de la República}{October 2017}{Montevideo, Uruguay}

\divider{}

\cvevent{Bender: Efficient Execution of Neural Networks in iOS}{Open Tech 2017}{September 2017}{Montevideo, Uruguay}

\divider{}

\cvevent{Humor Detection in Tweets}{9º Foro de Lenguas de ANEP}{December 2016}{Montevideo, Uruguay}

\cvevent{Is This a Joke? Detecting Humor in Spanish Tweets}{IBERAMIA 2016}{November 2016}{San José, Costa Rica}

\divider{}

\cvevent{Automatic Detection of False Friends}{\nth{2} Workshop of RITA Project}{March 2016}{Porto Alegre, Rio Grande do Sul, Brazil}

\divider{}

\cvevent{Humor Detection in Spanish Tweets}{Seminar of the Computer Science Department, Universidad de la República}{June 2015}{Montevideo, Uruguay}

\divider{}

\cvsection{Conferences \& Workshops Attended}

\cvevent{\nth{27} International Conference on Computational Linguistics (COLING 2018)}{Santa Fe Community Convention Center}{August 2018}{Santa Fe, New Mexico, USA}

\divider{}

\cvevent{\nth{5} Workshop on NLP for Similar Languages, Varieties and Dialects (VarDial 2018)}{COLING 2018, Santa Fe Community Convention Center}{August 2018}{Santa Fe, New Mexico, USA}

\divider{}

\cvevent{\nth{6} International Workshop on Natural Language Processing for Social Media (SocialNLP
2018)}{ACL 2018, Melbourne Convention and Exhibition Centre}{July 2018}{Melbourne, Australia}

\divider{}

\cvevent{\nth{56} Annual Meeting of the Association for Computational Linguistics (ACL 2018)}
{Melbourne Convention and Exhibition Centre}{July 2018}{Melbourne, Australia}

\divider{}

\cvevent{9º Foro de Lenguas de ANEP}{Instituto de Perfeccionamiento y Estudios Superiores}{December 2016}{Montevideo, Uruguay}

\divider{}

\cvevent{\nth{16} Ibero-American Conference on Artificial Intelligence (IBERAMIA 2016)}{Universidad Nacional de Costa Rica}{November 2016}{San José, Costa Rica}

\divider{}

\cvevent{\nth{2} Workshop of RITA Project --- RIch Text Analysis through Enhanced Tools based on Lexical Resources}{Universidade Federal do Rio Grande do Sul}{March 2016}{Porto Alegre, Rio Grande do Sul, Brazil}

\cvsection{References}

\ifpublic{}
  Available upon request.
\else
  \input{ref}
\fi

\divider{}

(Resume last update: Sept. \nth{6}, 2018)

\end{document}
