\documentclass[10pt,a4paper,academicons]{altacv}

\usepackage[T1]{fontenc}
\usepackage{microtype}
\usepackage{hyperref}
\usepackage[super]{nth}

\geometry{left=1cm,right=9cm,marginparwidth=6.8cm,marginparsep=1.2cm,top=1cm,bottom=1cm}

\setmainfont{Lato}

\hypersetup{colorlinks,linkcolor={cyan},citecolor={cyan},urlcolor={cyan}}

\definecolor{Vivid}{HTML}{000000}
\definecolor{SlateGrey}{HTML}{2E2E2E}
\definecolor{LightGrey}{HTML}{666666}
\colorlet{heading}{Vivid}
\colorlet{accent}{Vivid}
\colorlet{emphasis}{SlateGrey}
\colorlet{body}{LightGrey}

\newcommand{\googlescholar}[1]{\printinfo{\faGraduationCap}{#1}}
\newcommand{\stackoverflow}[1]{\printinfo{\faStackOverflow}{#1}}
\newcommand{\home}[1]{\printinfo{\faHome}{#1}}

\DeclareNameAlias{author}{given-family}

\addbibresource{main.bib}

\AtEveryBibitem{\clearfield{pages}}
\AtEveryBibitem{\clearfield{volume}}
\AtEveryBibitem{\clearfield{number}}
\AtEveryBibitem{\clearfield{doi}}
\AtEveryBibitem{\clearfield{issn}}
\AtEveryBibitem{\clearlist{publisher}}
\AtEveryBibitem{\clearlist{location}}

\makeatletter
\def\mblx@yearfrom{-1000}
\def\mblx@yearto{3000}

\newrobustcmd*{\setcheckfromtorange}[2]{%
  \def\mblx@yearfrom{#1}%
  \def\mblx@yearto{#2}%
}

\defbibcheck{fromto}{%
  \iffieldint{year}
    {\ifnumless{\thefield{year}}{\mblx@yearfrom}
       {\skipentry}
       {\ifnumgreater{\thefield{year}}{\mblx@yearto}
          {\skipentry}
          {}}}
    {\skipentry}}
\makeatother

\newif\ifpublic{}

\publictrue{}
%\publicfalse{}

\begin{document}

\name{Santiago Castro}
\tagline{PhD Student}
\personalinfo{%
  \email{\href{mailto:sacastro@umich.edu}{sacastro@umich.edu}}
  \location{Ann Arbor, MI, USA}
  \homepage{\href{https://santi.uy}{santi.uy}}
  \linkedin{\href{https://linkedin.com/in/santiagocastroserra}{santiagocastroserra}}
  \googlescholar{\href{https://scholar.google.com/citations?user=i2LNBfUAAAAJ}{Santiago Castro}}
  \github{\href{https://github.com/bryant1410}{bryant1410}}
  \twitter{\href{https://twitter.com/bryant1410}{bryant1410}}
  \stackoverflow{\href{https://stackoverflow.com/users/1165181/bryant1410}{bryant1410}}
}

\begin{fullwidth}

\makecvheader{}

\textit{Diligent Computer Scientist with research experience in the industry and academia.}

\end{fullwidth}

\cvsection[page1sidebar]{Education}

\cvevent{Ph.D.\ in Computer Science}{University of Michigan}{Sept. 2018 --- Present}{Ann Arbor, MI, USA}

{\small
\begin{itemize}
  \item Advisor: \href{https://web.eecs.umich.edu/~mihalcea/}{Rada Mihalcea}
  \item My main research interest is in holistically representing \textbf{Multimodal} sources (i.e., Language and Vision) for general-purpose applications, esp.\ for video tasks. I am also interested in Computational Humor.
  \item GPA: 4.0/4.0
\end{itemize}
}

\divider{}

\cvevent{Eng.\ in Computer Science}{Universidad de la República}{March 2010 --- May 2015}{Montevideo, Uruguay}

{\small
\begin{itemize}
  \item Extracurricular courses: Introduction to Deep Learning, Deep Learning applied to Natural
  Language Processing, Formal Grammars for Natural Language, General-Purpose Computing on
  Graphics Processing Units and Semantic Parsing.
  \item GPA rank: 4th/665
\end{itemize}
}

\cvsection{Selected Publications}

\href{https://scholar.google.com/citations?user=i2LNBfUAAAAJ}{26 publications, 725 citations, and h-index 12 in Google Scholar.}

\vspace{3mm}

\nocite{probing-clip,phenaki,fitclip,wildqa,fiber,whyact,haha-dataset,lifeqa,mustard,humor-detection}

{
\hypersetup{hidelinks}
\printbibliography[heading=none]
}

\cvsection[page2sidebar]{Research Work Experience}

\cvevent{Research Intern}{Netflix}{May--August 2023}{Los Gatos, CA, USA (remote)}

{\small
Project on Compositional Generalization of Vision-Language Models. Mentors: \href{https://www.linkedin.com/in/amirziai}{Amir Ziai}, \href{https://asaluja.github.io/}{Avneesh Saluja}, and \href{https://zhuoning.cc/}{Zhuoning Yuan}
}

\divider{}

\cvevent{Research Intern}{Google Brain}{May--August 2022}{USA (remote)}

{\small
Project on Video Representation Learning using Language. Mentors: \href{https://web.engr.illinois.edu/~mb2}{Mohammad Babaeizadeh} and \href{https://rubenvillegas.me/}{Ruben Villegas}
}

\divider{}

\cvevent{Research Intern}{Adobe Research}{May--August 2021}{San José, CA, USA (remote)}

{\small
Project on Video Representation Learning. Mentor: \href{https://fabiancaba.com/}{Fabian Caba}
}

\divider{}

\cvevent{Research Intern}{Netflix}{May--August 2020}{Los Gatos, CA, USA (remote)}

{\small
Project on Language and Vision Representation Learning through videos. Manager: \href{https://vinmisra.github.io/}{Vinith Misra}
}

\divider{}

\cvevent{Principal Research Engineer}{Xmartlabs}{September 2016 --- August 2018}{Montevideo, Uruguay}

{\small
I started a Machine Learning area within the company, highly focused on Mobile Computer Vision. It included the conception, design, and engineering leadership of the open-source project {\href{https://github.com/xmartlabs/Bender}{Bender}}, the first framework to run neural nets in real-time on an iPhone using the GPU.\@ After its release, competitors have subsequently emerged from big tech companies (i.e., Google’s TF Lite, Apple’s Core ML, Facebook’s Caffe2Go, and Baidu’s MDL). I also worked on other CV+ML projects for the Entertainment, Retail, and Agro industries.
}

\divider{}

\cvevent{Teaching \& Research Assistant (``Asistente --- Grado 2'')}{Universidad de la República}{March 2014 --- August 2018}{Montevideo, Uruguay}

\cvsection{Other Work Experience}

\cvevent{Android Engineer}{Xmartlabs}{March 2014 --- August 2016}{Montevideo, Uruguay}

\cvsection{Patents}

\cvevent{Adapting Pretrained Classification Models to Different Domains (pending)}{}{April 2022}{}

\cvsection{Invited Talks}

\cvevent{GPGPU in the context of Deep Learning}{GPGPU class, Universidad de la República}{June 2021}{Montevideo, Uruguay (remote)}

\divider{}

\cvevent{Mobile Machine Learning}{Machine Learning undergrad class, Universidad de la República}{May 2018}{Montevideo, Uruguay}

\divider{}

\cvevent{Mobile Machine Learning with Bender}{Deep Learning Applied to Computer Vision graduate class, Universidad de la República}{October 2017}{Montevideo, Uruguay}

\divider{}

\cvevent{Bender: Efficient Execution of Neural Networks in iOS}{Open Tech 2017}{September 2017}{Montevideo, Uruguay}

\divider{}

\cvevent{Humor Detection in Tweets}{9º Foro de Lenguas de ANEP}{December 2016}{Montevideo, Uruguay}

\divider{}

\cvevent{Humor Detection in Spanish Tweets}{CS Department Seminar, Universidad de la República}{June 2015}{Montevideo, Uruguay}

\cvsection{Press}

\cvevent{\href{https://medium.com/@akatuma/finding-and-lifting-up-diamonds-in-the-rough-shaping-the-next-generation-of-ai-researchers-ed7c04432362}{Finding and Lifting Up Diamonds in the Rough: Shaping the Next Generation of AI Researchers}}{Rada Mihalcea's Medium Blog}{January 2024}{}

{\small
``\textit{I am giving the ``then and now'' profile of one of my current PhD students, Santiago Castro.}''
}

\divider{}

\cvevent{\href{https://netflixtechblog.com/building-in-video-search-936766f0017c}{Building In-Video Search}}{Netflix Technology Blog}{November 2023}{}

{\small
I'm a coauthor of this technology.
}

\divider{}

\cvevent{\href{https://open.spotify.com/episode/4ORMa33bQPTEuUlVIx7mWw?si=N8tSrT3ASQycxxINodJ0mQ}{BD61 - Santiago Castro, estudiante de Doctorado}}{Podcast Bigdatéame}{February 2021}{}

{\small
``\textit{Santiago Castro, PhD Student}.''
}

\divider{}

\cvevent{\href{https://www.lr21.com.uy/comunidad/391358-nicosanti-ganador-del-concurso-de-la-fundacion-de-cultura-informatica}{Nicosanti ganador del concurso de la Fundación de Cultura Informática}}{LaRed21}{December 2009}{Montevideo, Uruguay}

{\small
``\textit{NicoSanti team winner of the coding challenge}'', organized by Fundación de Cultura Informática and Microsoft Uruguay.
}

\cvsection{References}

\ifpublic{}
  {\small Available upon request.}
\else
  \input{ref}
\fi

\divider{}

{\small
(Resume last update: \today)
}

\end{document}
