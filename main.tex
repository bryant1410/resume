\documentclass[10pt,a4paper,academicons]{altacv}

\usepackage[T1]{fontenc}
\usepackage[utf8]{inputenc}
\usepackage{microtype}
\usepackage{hyperref}
\usepackage[super]{nth}

\geometry{left=1cm,right=9cm,marginparwidth=6.8cm,marginparsep=1.2cm,top=1cm,bottom=1cm}

\setmainfont{Lato}

\hypersetup{colorlinks,linkcolor={cyan},citecolor={cyan},urlcolor={cyan}}

\definecolor{Vivid}{HTML}{000000}
\definecolor{SlateGrey}{HTML}{2E2E2E}
\definecolor{LightGrey}{HTML}{666666}
\colorlet{heading}{Vivid}
\colorlet{accent}{Vivid}
\colorlet{emphasis}{SlateGrey}
\colorlet{body}{LightGrey}

\newcommand{\googlescholar}[1]{\printinfo{\faGraduationCap}{#1}}
\newcommand{\stackoverflow}[1]{\printinfo{\faStackOverflow}{#1}}
\newcommand{\home}[1]{\printinfo{\faHome}{#1}}

\addbibresource{main.bib}

\AtEveryBibitem{\clearfield{pages}} 

\newif\ifpublic{}

\publictrue{}
%\publicfalse{}

\begin{document}

\name{Santiago Castro}
\tagline{PhD Student}
\personalinfo{%
  \email{\href{mailto:sacastro@umich.edu}{sacastro@umich.edu}}
  \location{Ann Arbor, MI, USA}
  \homepage{\href{https://santi.uy}{santi.uy}}
  \googlescholar{\href{https://scholar.google.com/citations?user=i2LNBfUAAAAJ}{Santiago Castro}}
  \github{\href{https://github.com/bryant1410}{bryant1410}}
  \twitter{\href{https://twitter.com/bryant1410}{bryant1410}}
  \linkedin{\href{https://linkedin.com/in/santiagocastroserra}{santiagocastroserra}}
  \stackoverflow{\href{https://stackoverflow.com/users/1165181/bryant1410}{bryant1410}}
}

\begin{fullwidth}

\makecvheader{}

Diligent Computer Scientist with research experience in the industry and academia.

\end{fullwidth}

\cvsection[page1sidebar]{Education}

\cvevent{Ph.D.\ in Computer Science}{University of Michigan}{Sept. 2018 --- Present}{Ann Arbor, MI, USA}

\begin{itemize}
  \item Advisor: \href{https://web.eecs.umich.edu/~mihalcea/}{Rada Mihalcea}
  \item My main research interest is in holistically representing \textbf{Multimodal} sources (i.e., Language and Vision) for general-purpose applications, esp. for video tasks. Apart from this, I am interested in Computational Humor.
  \item GPA: 4.0/4.0
\end{itemize}

\divider{}

\cvevent{B.E.\ in Computer Science}{Universidad de la República}{March 2010 --- May 2015}{Montevideo, Uruguay}

\begin{itemize}
  \item Extracurricular courses: Introduction to Deep Learning, Deep Learning applied to Natural
  Language Processing, Formal Grammars for Natural Language, General-Purpose Computing on
  Graphics Processing Units and Semantic Parsing.
  \item GPA rank: 4th/665
\end{itemize}

\cvsection{Publications}

\href{https://scholar.google.com/citations?user=i2LNBfUAAAAJ}{302 citations and h-index 9 in Google Scholar.}

\vspace{3mm}

\nocite{*}

{
\hypersetup{hidelinks}

\printbibliography[heading=pubtype,title={\printinfo{\faGroup}{Conference Proceedings}},type=inproceedings]

\pagebreak

\cvsection[page2sidebar]{Publications (continued)}

\printbibliography[heading=pubtype,title={\printinfo{\faBook}{Journal Articles}},type=article]

\printbibliography[heading=pubtype,title={\printinfo{\faGraduationCap}{Theses}},type=thesis]

\printbibliography[heading=pubtype,title={\printinfo{\faGroup}{Workshop Proceedings}},type=incollection]

\printbibliography[heading=pubtype,title={\printinfo{\faAsterisk}{Non Refereed}},type=misc]
}

\cvsection{Research Work Experience}

\cvevent{Research Intern}{Google Brain}{May 2022 --- August 2022}{USA (remotely)}

Project on Video Representation Learning using Language. Mentors: \href{https://web.engr.illinois.edu/~mb2}{Mohammad Babaeizadeh} and \href{https://rubenvillegas.me/}{Ruben Villegas}

\divider{}

\cvevent{Research Intern}{Adobe Research}{May 2021 --- August 2021}{San José, CA, USA (remotely)}

Project on Video Representation Learning. Mentor: \href{https://fabiancaba.com/}{Fabian Caba}

\divider{}

\cvevent{Research Intern}{Netflix}{May 2020 --- August 2020}{Los Gatos, CA, USA (remotely)}

Project on Language and Vision Representation Learning through videos. Manager: \href{https://vinmisra.github.io/}{Vinith Misra}

\divider{}

\cvevent{Principal Research Engineer}{Xmartlabs}{September 2016 --- August 2018}{Montevideo, Uruguay}

\begin{itemize}
  \item I started a Machine Learning area within the company, highly focused on Mobile Computer Vision. It included the conception, design, and engineering leadership of the open-source project {\href{https://github.com/xmartlabs/Bender}{Bender}}, the first framework to run neural nets in real-time on an iPhone using the GPU.\@ After its release, competitors have subsequently emerged from big tech companies (i.e., Google’s TF Lite, Apple’s Core ML, Facebook’s Caffe2Go, and Baidu’s MDL). I also worked on other CV+ML projects for the Entertainment, Retail, and Agro industries.
\end{itemize}

\divider{}

\cvevent{Teaching \& Research Assistant (``Asistente --- Grado 2'')}{Universidad de la República}{March 2014 --- August 2018}{Montevideo, Uruguay}

\cvsection{Other Work Experience}

\cvevent{Android Engineer}{Xmartlabs}{March 2014 --- August 2016}{Montevideo, Uruguay}

\cvsection{Talks}

\cvevent{GPGPU in the context of Deep Learning}{GPGPU class, Universidad de la República}{June 2021}{Montevideo, Uruguay (remotely)}

\divider{}

\cvevent{Overview of the HAHA Task: Humor Analysis based on Human Annotation}{IberLEF @ SEPLN 2019}{September 2019}{Bilbao, Spain}

\divider{}

\cvevent{A High Coverage Method for Automatic False Friends Detection for Spanish and Portuguese}{VarDial 2018 @ COLING 2018}{August 2018}{Santa Fe, New Mexico, USA}

\divider{}

\cvevent{A Crowd-Annotated Spanish Corpus for Humor Analysis}{SocialNLP 2018 @ ACL 2018}{July 2018}{Melbourne, Australia}

\divider{}

\cvevent{Mobile Machine Learning}{Machine Learning undergraduate course, Universidad de la República}{May 2018}{Montevideo, Uruguay}

\divider{}

\cvevent{Mobile Machine Learning with Bender}{Deep Learning Applied to Computer Vision graduate course, Universidad de la República}{October 2017}{Montevideo, Uruguay}

\divider{}

\cvevent{Bender: Efficient Execution of Neural Networks in iOS}{Open Tech 2017}{September 2017}{Montevideo, Uruguay}

\divider{}

\cvevent{Humor Detection in Tweets}{9º Foro de Lenguas de ANEP}{December 2016}{Montevideo, Uruguay}

\divider{}

\cvevent{Is This a Joke? Detecting Humor in Spanish Tweets}{IBERAMIA 2016}{November 2016}{San José, Costa Rica}

\divider{}

\cvevent{Automatic Detection of False Friends}{\nth{2} Workshop of RITA Project}{March 2016}{Porto Alegre, Rio Grande do Sul, Brazil}

\divider{}

\cvevent{Humor Detection in Spanish Tweets}{Seminar of the Computer Science Department, Universidad de la República}{June 2015}{Montevideo, Uruguay}

\cvsection{Press}

\cvevent{\href{https://www.lr21.com.uy/comunidad/391358-nicosanti-ganador-del-concurso-de-la-fundacion-de-cultura-informatica}{Nicosanti ganador del concurso de la Fundación de Cultura Informática}}{LaRed21}{December 2009}{Montevideo, Uruguay}

``\textit{NicoSanti team winner of the coding challenge}'', organized by Fundación de Cultura Informática and Microsoft Uruguay.

\cvsection{References}

\ifpublic{}
  Available upon request.
\else
  \input{ref}
\fi

\divider{}

(Resume last update: \today)

\end{document}
